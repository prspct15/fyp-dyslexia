\begin{abstract}
The project aims to develop a machine learning model that can be applied in a system, to detect potential dyslexia through handwriting images. The approach taken was using the LeNet-5 model. The first objective in this study is to identify the features of dyslexia handwriting in images. This is achieved by performing a thorough literature review on the features of dyslexia handwriting, which spans from various journals, research articles, and books published by academics and researchers all over the world. The second objective of the study is to develop a machine learning model that distinguish dyslexic or non-dyslexic samples. This is achieved by using a modified version of the LeNet-5 model, which is a convolutional neural network (CNN) model that is used to classify handwritten digits. The model is modified to be able to classify dyslexic or non-dyslexic samples. The model is then trained using the dataset that was pre-processed. The third objective of the study is to evaluate the performance of the machine learning model. This is achieved by evaluating the performance of the model using the accuracy, precision, recall, and F1-score metrics. The model is also evaluated using the confusion matrix and classification report. The model is also tested using a sample of the dataset that was not used in the training process. However, the model is found to be overfitting, which can be related to the publicly available dataset, which is unoptimized for the model. In terms of recommendations, the first recommendation is to acquire a better dataset. The dataset used in this study is heavily imabalanced, which is not optimized for the model. This is evident in the overfitting of the model, where the model is not able to generalize well. Once the model is optimized, the model can be deployed in a system, where the system can be used to detect dyslexia in children. The system can be used to detect dyslexia in children at a young age, as early diagnosis of dyslexia can allow the children to be taught to read and write properly, and will allow them to be able to perform well in school, on par with their peers.
\end{abstract}
