\chapter{LITERATURE REVIEW}
\label{ch:litreview}


\section{Dyslexia}

\subsection{Definition and Prevalence}

Dyslexia can be defined in so many ways. According to Merriam-Webster, dyslexia is defined as "a variable often familial learning disability involving difficulties in acquiring and processing language that is typically manifested by a lack of proficiency in reading, spelling, and writing" \parencite{mw:dyslexia}. The word's history dates back to 1888, where it is initially defined as an ``impairment in the ability to read due to a brain injury''. It is borrowed from French and German, where both it is written as \emph{dyslexie}. 

Estimates for the prevalence of dyslexia suggest that it affects around 5 to 17\% \parencite{Ramli2020} of the global population, marking it as one of the most common learning disorders. However, these statistics are subject to considerable variance due to the numerous definitions of dyslexia, the diverse methodologies employed for its diagnosis, and the sampling of different populations in research studies.

Despite these disparities in prevalence rates, what remains consistent is the universal recognition of dyslexia's significant impact on individuals' educational journeys, their professional progress, and even their day-to-day lives. It is also evident that dyslexia do not discriminate between genders alike, according to \textcite{Guerin1993DyslexicSA}.


\subsection{Causes and Risk Factors}
The causes of dyslexia are complex and multifaceted, involving an intricate interplay of genetic, neurobiological, and environmental factors. In fact, some of the causes are yet to be fully understood. 

Genetic researches have identified several genomic regions that have been linked to dyslexia, including chromosomes 6 and 18 \parencite{Francks2002, Schumacher2007}. Advancements in research has also found identifying candidate genes for dyslexia, which includes Dyslexia Susceptibility 1 Candidate Gene 1, Roundabout Guidance Receptor 1, KIAA0319, and Doublecortin Domain Containing 2 \parencite{Paracchini2007}. 

From a neurobiological standpoint, structural and functional differences have been identified in the brains of individuals with dyslexia. Neuroimaging studies have identified differences in brain structure and function in individuals with dyslexia, including disruptions in left hemisphere posterior brain systems and increased reliance on frontal lobe circuits \parencite{Kearns2019, Norton2015}.

Another research has found that dyslexia is highly heritable and displays polygenic transmission, and adult neuroimaging studies have found structural, functional, and physiological changes in the parieto-occipital and occipito-temporal regions, and in the inferior frontal gyrus, in adults with dyslexia \parencite{SorianoFerrer2017}. 

Another research also reviewed evidence of autopsy and structural imaging studies and found consistent evidence of symmetry of the planum temporale, thalamus, and cortical malformations in individuals with dyslexia \parencite{Wajuihian2011}.

Environmental factors further contribute to the manifestation and development of dyslexia. found that there is a substantial social-cultural bias in the delineation of literacy skills and in the definitions of reading disabilities, and suggests that phonological deficits should be emphasized as the core component in defining dyslexia \parencite{Samuelsson2003}.

\newpage
\subsection{Characteristics}
Dyslexia encompasses a variety of symptoms, which generally become apparent once a child starts school and is confronted with the challenges of learning to read and write. Common characteristics of dyslexia include difficulty with phonological skills, low accuracy and fluency of reading, poor spelling, and/or rapid visual-verbal responding \parencite{Roitsch2019}. 

These signs can vary in intensity and nature among individuals, contributing to the spectrum of dyslexia manifestations. Additionally, dyslexia can extend to erratic eye movements during reading and other tasks \parencite{Pavlidis1981}.

\newpage
\subsection{Impact on Writing and Handwriting}
Dyslexia's impact extends noticeably to an individual's handwriting, with various studies indicating that individuals with dyslexia often grapple with handwriting fluency, neatness, and speed. A study found that children with dyslexia struggle with the graphomotor aspects of writing and are more impacted by the graphic complexity of words than typically developing children \parencite{Gosse2020}. 

Another study has also found that spelling ability influences the rate of handwriting production in children with dyslexia, and that productivity relies on spelling capabilities \parencite{Sumner2014}.

Writing speed also varies within children with dyslexia, as a study found that handwriting speed in Chinese children with dyslexia is related to deficits in rapid automatic naming, saccadic efficiency, and visual-motor integration \parencite{ChengLai2013}. 

Besides that, spelling deficits associated with dyslexia affect the dynamics of the interaction between central and peripheral processes and the level of anticipation that can be observed in word spelling in the context of a sentence to dictation task \parencite{SurezCoalla2020}.

\newpage
\subsection{Current Diagnostic Methods}
The diagnosis of dyslexia usually involves a holistic evaluation of the individual's academic performance, cognitive and linguistic skills, and developmental history. 

One study discusses the diagnostic assessment of dyslexia, which consists of standardized reading and spelling tests, evaluation of psychological state, and additional information from parents and teachers \parencite{SchulteKrne2010}. The diagnostic procedure may incorporate a battery of tests to assess reading, spelling, and writing skills. 

However, in recent times, studies have discovered advancements in diagnostic procedures. One research proposes a diagnostic method based on involuntary neurophysiological responses to auditory stimuli, using Electroencephalogram signals to analyze temporal behavior and spectral content \parencite{Ortiz2020}. 

Another study has reviewed different technology-based approaches for dyslexia detection, including eye-tracking and Electroencephalogram devices, and statistical or machine learning algorithms \parencite{Jankovic2022}.

\newpage
\subsection{Interventions and Treatments}
Even though there is no known cure for dyslexia, early assessment and intervention, paired with supportive teaching strategies, can significantly improve success rates for individuals with dyslexia. These interventions often include multi-sensory, structured language programs that explicitly teach phonics, morphology, syntax, and semantics. 

A study has found that multisensory, phonological, and cognitive training methods can be used to improve literacy and cognitive deficits among children with dyslexia in Malaysia \parencite{Anis2018}. 

Besides that, in Brazil, there are four key themes of interventions for dyslexia, which includes phonological-based intervention, computerized technology, auditory processing training, and visuomotor skills training \parencite{Signor2020}.

However, on a national level, governments have to make strides in dyslexia intervention. Study suggests that early identification of children at risk of dyslexia followed by evidence-based interventions is a realistic aim for practitioners and policy-makers 
\parencite{Snowling2012}. 

\newpage
\subsection{Challenges and Limitations of Current Practices}
Current practices in dyslexia diagnosis and intervention, despite their effectiveness, present several challenges. 

For one, the cognitive approach alone is not sufficient to address dyslexia, and that a combination of cognitive psychology, connectionism, and behaviorism is necessary \parencite{Tnnessen1999}.

A local study also finds that there is a lack of comprehensive studies that combine interventions for both cognitive functions and literacy deficits \parencite{Anis2018}. 

Besides that, while appropriate instruction can help at-risk readers become accurate and fluent, intensive remedial interventions have been less effective in closing the fluency gap. \parencite{Alexander2004}.


